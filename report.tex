\documentclass[10pt]{article}
\usepackage[letterpaper]{geometry}
\usepackage[spacing,kerning]{microtype}

\usepackage{graphicx}
\usepackage{amsmath}

\newcommand{\norm}[1]{\left|\!\left|\,{#1}\,\right|\!\right|}

\title{MIPSRAY: A MIPS Ray Tracer\\\Large EECS 314: Computer Architecture}
\author{Josh Lee, Gary Doran}
\date{April 29, 2008}

\begin{document}
\begin{center}
\LARGE MIPSRAY: A MIPS Ray Tracer \\
\Large EECS 314: Computer Architecture \\
\medskip
\large Josh Lee, Gary Doran \\
April 29, 2008
\end{center}

\section{Problem Statement}

The objective of the MIPSRAY Project was to develop a Ray Tracer
using the MIPS Assembly Language. Ray tracing is a technique for
rendering images of three-dimensional scenes that models the physical
way that rays of light behave. As most light rays do not end up
reaching the camera, simulating the path of every ray of light would
be computationally infeasible. Instead, the algorithm shoots rays
out of the camera into the scene, then simulates the reflections and
refractions in reverse.

The actual ray tracing algorithm is relatively straightforward, given a
camera (specified with vectors for its position and the image frame),
a scene full of objects (e.g. spheres, planes), and light sources.
For each pixel in the frame, a ray is generated that starts at the
camera and passes through that pixel into the scene. The intersection
of the camera ray with each object in the scene is computed, and the
one nearest the camera is chosen. If there are no hits, the pixel is
assigned a background color. If the ray hits an object, a shadow ray is
shot towards each light source from the hit to determine whether (and
to what extent) the surface is illuminated. Furthermore, a reflected
ray is generated at the point of intersection, and the algorithm is
called recursively on this ray. In this case, the color of the pixel
is determined by combining the color of the object hit, the degree of
illumination, and color returned by the reflected ray. The process just
described is performed for every pixel in the frame to produce an image
of the scene.

\section{Major Challenges}

The major challenges of the MIPSRAY Project arose from attempting to
implement various aspects of the Ray Tracing algorithm. First of all,
the appropriate data structures had to be selected to represent various
objects in the scene. Because it is relatively easy to mathematically
manipulate vectors, three-vectors of double precision floating point
numbers were commonly used. Thus, rays became a pair of vectors, one
specifying the ray's origin and another unit vector specifying the
direction of the ray. Objects such as a sphere were primarily specified
by a vector indicating the sphere's center and a double for its radius,
along with other information pertaining to the sphere's color (another
vector) and reflectivity. Light sources were represented by a vector
for position and a vector for color. Because random access of the light
sources and scene objects was not necessary, linked lists were used to
store these objects. Accordingly, the first word of each object/light
source was either null or a pointer to the next object/light source. Of
course, along with these data structures came the challenge of creating
procedures to implement various operations on these data structures such
as vector operations or finding an intersection between a ray and an
object.

Other major challenges arose from more practical issues related to
implementation of the Ray Tracing algorithm. For example, the generated
image needed to be written to the hard disk in the form of a bitmap
(BMP) image file. This process required knowledge of both the BMP file
format, as well as the assembly instructions required to create, open
and write to the file. In addition, some basic mathematic knowledge was
required to implement certain aspects of the algorithm. For example,
basic vector arithmetic and the ability to find roots of polynomials
were required to find the intersection of a ray with a sphere. The
recursive nature of the algorithm required the stack to be used, which
added a small challenge to the implementation, particularly due to
booking concerning memory offset values. Furthermore, since it was
difficult to test certain parts of the program individually due to
both the nature of the algorithm and of assembly language, debugging
was difficult while meaningless output files were produced. Only after
recognizable (albeit incorrect) outputs were produced did debugging
become much less challenging.

\section{Key Components}

\subsection{vectors.s and roots.s}

The first key component of the implementation was the set of
mathematical tools needed to manipulate various data structures and
calculate intersections. A set of procedures, located in the file
``vectors.s'' were written to perform vector operations, including
addition, subtraction, dot product, and scalar multiplication. In
addition, several procedures calculate values such as the magnitude
of a vector, or return a unit vector in the direction of a given
vector. These procedures were used throughout the program. Another
file, ``roots.s'' houses procedures that find the roots of polynomials
(linear and quadratic equations). These procedures were used primarily
to calculate the intersections of rays with objects in the scene. The
main challenges with implementing the root finder involved considering
all possible cases for a given equation—an equation can have either
none, one, or two roots or a quadratic equation could actually be a
linear equation if the coefficient of the degree-two term is zero. The
procedures in ``vectors.s'' and ``roots.s'' provide a toolbox that is
used heavily throughout the rest of the ray tracing program.

\subsection{shapes.s}

Another key component of the ray tracer required finding points of
intersection, reflections, etc. involving rays and objects. The
procedures in the ``shapes.s'' file handle these operations.
Because intersections are found differently with different shapes,
main procedures ``intersect,'' ``reflect,'' and ``normal'' call
the appropriate shape-specific procedures depending on the type
of object they are passed as an argument. The intersect procedure
returns the distance from the origin of a ray to the first point a ray
intersects with an object (and specifies whether such an intersection
exists).

\paragraph{Spheres}

Given a ray specified by $\vec{s}$ and $\vec{d}$ (source
and direction), and a sphere specified by $\vec{c}$ and $r$ (center
and radius), the intersection can be found with some basic algebra.
First, assume that the ray intersects with the circle at a distance
$t$ away from its source. Then $\vec{s} + t\vec{d}$ will be the point
of intersection. By the definition of a sphere, the distance squared
between this point and the center of the circle will be $r^2$. So
$\norm{\vec{s}+t\vec{d}-\vec{c}}^2 = r^2$. We can write this formula as:
\begin{align*}
& \norm{(\vec{s}-\vec{c})+t\vec{d}}=r^2 \\
& \left((\vec{s}-\vec{c})+t\vec{d}\right)\cdot\left((\vec{s}-\vec{c})+t\vec{d}\right) = r^2 \\
& (\vec{d}\cdot\vec{d})t^2 + 2\vec{d}\cdot(\vec{s}-\vec{c})t+\left(\norm{\vec{s}-\vec{c}}^2-r^2\right) = 0
\end{align*}
This quadratic equation can then be solved for $t$ to find the distance
to the closest intersection. The equation will have no real roots if no
such intersection exists, and only negative roots if the intersection
occurs behind the ray---in both cases, the ray misses the object
entirely.

\paragraph{Planes}

Given a similar ray, and a plane specified by a point
$\vec{p}$ on the plane and a unit normal to the plane $\vec{n}$, an
intersection between a ray and a plane can be found. At the point of
intersection, $\vec{s}+t\vec{d} = \vec{p}+\vec{v}$, where $\vec{v}$ is
some vector that lies on the plane (i.e. is parallel to the plane).
Since $\vec{v}$ and $\vec{n}$ are necessarily at right angles to one
another, $\vec{v}\cdot\vec{i}=0$. Using this fact, the equation above
can be rewritten as:
\begin{align*}
& \vec{n}\cdot(\vec{s}+t\vec{d}) = \vec{n}\cdot(\vec{p}+\vec{v}) \\
& \left(\vec{n}\cdot\vec{s}\right) + (\vec{n}+\vec{d})t = (\vec{n}\cdot\vec{p}) + 0 \\
& (\vec{n}\cdot\vec{d})t + \vec{n}\cdot(\vec{s}-\vec{p}) = 0
\end{align*}
This linear equation can be solved for $t$ as with the sphere to find
the distance to the closest intersection with a plane. In addition to
finding intersection points, ``shapes.s'' also contains a procedure
to return the unit normal vector to a point of intersection. This
procedure is useful for calculating the intensity of light at a given
point on an object. Furthermore, ``shapes.s'' contains a procedure
that generates a reflected ray from an intersection between a point
and an object. To do this, it uses the fact that the starting point
is the point of intersection and the direction is given by $\vec{r} =
\vec{d}-2(\vec{n}\cdot\vec{d})\vec{n}$, where $\vec{n}$ is the unit
normal to intersection and $\vec{d}$ is the direction of the original
ray.

\smallskip
The starting point of the reflected ray is actually set a small
distance along the direction of $\vec{r}$, because otherwise subsequent
calls to the procedure ``intersect'' would return a value of zero
(since it would think that the ray intersected with the object at
its origin). Finally, the ``nearestHit'' procedure loops through
each object in the scene with a given ray and finds the distance to
intersection with each object. It determines whether a hit occurs
with any object, and returns the object that is first hit by the ray.
All of these procedures are critical components of the ray tracer
implementation.

\subsection{bmp.s}

To save images to a bitmap file, a module was required to first write
the bitmap file header, then output each pixel as it was rendered.
As the BMP files were only written, not read, a full-featured image
loader was not necessary. A BMP file begins with a fixed-length header
specifying, among other fixed values, the image's width, height, and
size in bytes. The width and height are specified in the scene file,
so the bitmap module can compute them at the beginning, then write the
entire header out to the file. The ``bmp.write\_header'' procedure
fills in the header variables based on the width and height given. The
``bmp.write\_pixel'' procedure takes three bytes, one for each of red,
green, and blue, and writes them to the file. Each row of a BMP file
ends with up to 3 bytes of padding for word alignment purposes, so this
procedure also increments a counter to see if the end of the row is
reached.

\subsection{main.s}

The last major component of the MIPSRAY project was the actual
implementation of the Ray Tracing Algorithm. The algorithm was
implemented in the file ``main.s'' in three major procedures. The
first procedure, ``main,'' create an image file, loops through pixels
in the frame, and generates a ray passing through each pixel. It then
calls the ``raytrace'' procedure on this ray, and writes the returned
color to the image file. The ``raytrace'' procedure is recursive,
and returns immediately if the proper level of recursion is reached.
Otherwise, it looks for the nearest hit object. If no object is hit by
the ray, a background color (e.g. black) is returned. If an object is
hit, the ``shadow'' procedure calculates the intensity of the color at
that point given light sources and obstructions. Next, a reflected ray
is generated and ``raytrace'' is called recursively on this ray. The
contributions from the light and the reflected ray are added together,
and this value is returned. The ``shadow'' procedure works by looping
through each light source in the scene and determining whether another
object obstructs the path to the light source. If so, then only a basic,
ambient amount of light is illuminating the point of intersection.
Otherwise, the intensity of the light from that light source is given by
the cosine of the angle between the direction to the light source and
the unit normal to the intersection. This value can be found using a dot
product. The contributions are summed over all lights, and this value
is returned to the ``raytrace'' procedure. All of these components are
necessary to implement the Ray Tracing Algorithm.

\section{Integration of Components}

Because various components of the ray tracer contain different sets of tools, they are relatively easy to integrate.  Procedures in ``main.s'' depend directly on functions supplied by the files ``shapes.s,'' ``bmp.s,'' and ``vectors.s.''  The procedures in ``shapes.s'' also rely heavily on procedures in ``roots.s'' and ``vectors.s.''  Major challenges in integration arose during debugging, when it was sometimes impractical to test functions of ``lower-level'' procedures except by their role in ``higher-level'' procedures in ``main.s.''  In such cases, identifying the root cause of the bug became difficult.  Otherwise, integration of components was relatively straightforward.

\section{User Interface}

%\begin{figure}
%\centering
%\includegraphics[width=\textwidth]{output45}
%\caption{Example scene with multiple light sources, spheres, and
%reflections.}
%\end{figure}

The input and output to the ray tracer are primarily the scene and the
image file, respectively. Various aspects of the scene are specified in
a file called ``scene.s.'' The scene file specifies the output filename,
the dimensions of the image, the camera, the objects, the light sources,
and the shading parameters. The two parameters, ``ambient'' and
``diffuse'', respectively determine how much of an object's illumination
comes from external light sources, and how much it will be seen when it
is completely in shadow. The background color, used when no object is
hit by a ray, is also specified here.

To render a scene, all of the source files (bmp, vectors, roots, shapes,
main, scene) are loaded into the SPIM simulator. When the program is
executed, the scene is automatically rendered and saved to the scene's
filename.

\end{document}
